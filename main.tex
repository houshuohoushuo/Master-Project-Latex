%%%%%%%%%%%%%%%%%%%%%%%%%%%%%%%%%%%%%%%%%
% McMaster Masters/Doctoral Thesis 
% LaTeX Template
% Version 2.2 (11/23/15)
%
% This template has been downloaded from:
% http://www.LaTeXTemplates.com
% Then subsequently from http://www.overleaf.com
%
% Version 2.0 major modifications by:
% Vel (vel@latextemplates.com)
%
% Original authors:
% Steven Gunn  (http://users.ecs.soton.ac.uk/srg/softwaretools/document/templates/)
% Sunil Patel (http://www.sunilpatel.co.uk/thesis-template/)
%
% Modified to McMaster format by Benjamin Furman (contact: https://www.xenben/com; Most up 
% to date template at https://github.com/benjaminfurman/McMaster_Thesis_Template, 
% occasionally updated on Overleaf template page)
%
% License:
% CC BY-NC-SA 3.0 (http://creativecommons.org/licenses/by-nc-sa/3.0/)
%
%%%%%%%%%%%%%%%%%%%%%%%%%%%%%%%%%%%%%%%%%

%----------------------------------------------------------------------------------------
% DOCUMENT CONFIGURATIONS
%----------------------------------------------------------------------------------------

\documentclass[
12pt, % The default document font size, options: 10pt, 11pt, 12pt
oneside, % Two side (alternating margins) for binding by default, uncomment to switch to one side
english, % other languages available
onehalfpacing, % Single line spacing, alternatives: onehalfspacing or doublespacing
%draft, % Uncomment to enable draft mode (no pictures, no links, overfull hboxes indicated)
%nolistspacing, % If the document is onehalfspacing or doublespacing, uncomment this to set spacing in lists to single
%liststotoc, % Uncomment to add the list of figures/tables/etc to the table of contents
%toctotoc, % Uncomment to add the main table of contents to the table of contents
]{McMasterThesis} % The class file specifying the document structure


%----------------------------------------------------------------------------------------
% Import packages here
%----------------------------------------------------------------------------------------
\usepackage[utf8]{inputenc} % Required for inputting international characters
\usepackage[T1]{fontenc} % Output font encoding for international characters

\usepackage{lmodern} % could change font type by calling a different package
\usepackage{lastpage} % count pages
\usepackage{siunitx} % for scientific units (micro-liter, etc)
\setcounter{tocdepth}{2} % so that only section and sub sections appear in Table of Contents. Remove or set depth to 3 to include sub-sub-sections

%----------------------------------------------------------------------------------------
% Handling Citations
%----------------------------------------------------------------------------------------
\usepackage[backend=biber, giveninits=true, doi=false, natbib=true, url=false, eprint=false, style=authoryear, sorting=nyt, maxcitenames=2, maxbibnames=99, uniquename=false, uniquelist=false, dashed=false]{biblatex} % can change the maxbibnames to cut long author lists to specified length followed by et al., currently set to 99.
\DeclareFieldFormat[article,inbook,incollection,inproceedings,patent,thesis,unpublished]{title}{#1\isdot} % removes quotes around title
\renewbibmacro*{volume+number+eid}{%
  \printfield{volume}%
%  \setunit*{\adddot}% DELETED
  \printfield{number}%
  \setunit{\space}%
  \printfield{eid}}
\DeclareFieldFormat[article]{number}{\mkbibparens{#1}}
%\renewcommand*{\newunitpunct}{\space} % remove period after date, but I like it. 
\renewbibmacro{in:}{\ifentrytype{article}{}{\printtext{\bibstring{in}\intitlepunct}}} % this remove the "In: Journal Name" from articles in the bibliography, which happens with the ynt 
\renewbibmacro*{note+pages}{%
    \printfield{note}%
    \setunit{,\space}% could add punctuation here for after volume
    \printfield{pages}%
    \newunit}    
\DefineBibliographyStrings{english}{% clears the pp from pages
  page = {\ifbibliography{}{\adddot}},
  pages = {\ifbibliography{}{\adddot}},
} 
\DeclareNameAlias{sortname}{last-first}
\renewcommand*{\nameyeardelim}{\addspace} % remove comma in text between name and date
\addbibresource{Bibliography.bib} % The filename of the bibliography
\usepackage[autostyle=true]{csquotes} % Required to generate language-dependent quotes in the bibliography

% you'll have to play with the citation styles to resemble the standard in your field, or just leave them as is here. 
% or, if there is a bst file you like, just get rid of all this biblatex stuff and go back to bibtex. 

%----------------------------------------------------------------------------------------
% Collect all your header information from the chapters here, things like acronyms, custom commands, necessary packages, etc. 
%----------------------------------------------------------------------------------------
\usepackage{parskip} %this will put spaces between paragraphs
\setlength{\parindent}{15pt} % this will create and indent on all but the first paragraph of each section. 
% should maybe change to glossaries package
\usepackage{acro}
\DeclareAcronym{est}{
	short = EST,
	long  = expressed sequence tags
}

\DeclareAcronym{Xl}{
	short = \textit{X.~laevis},
	long  = \textit{Xenopus~laevis}
}
\DeclareAcronym{Xg}{
	short = \textit{X.~gilli},
	long  = \textit{Xenopus~gilli}
}

\usepackage{etoolbox}
\preto\chapter{\acresetall} % resets acronyms for each chapter

\usepackage{xspace} %helps spacing with custom commands. 
\newcommand{\oddname}{{\sc SoME goOfY LonG ThiNg With an AwkWarD NAme}\xspace}


\usepackage{pgfplotstable} % a much better way to handle tables
\pgfplotsset{compat=1.12}

% \usepackage{float} % if you need to demand figure/table placement, then this will allow you to use [H], which demands a figure placement. Beware, making LaTeX do things it doesn't want may lead to oddities.  


%%%%
% LINK COLORS
% You can control the link colors at the end of the McMasterThesis.cls file. There is also a true/false option there to turn off all link colors.  
%%%%


%----------------------------------------------------------------------------------------
%	THESIS INFORMATION
%----------------------------------------------------------------------------------------

\thesistitle{Breast Cancer Specimen Classification} % Your thesis title, print it elsewhere with \ttitle
\supervisor{Dr. Nedialkov \textsc{Ned}} % Your supervisor's name, print it elsewhere with \supname
\examiner{} % Your examiner's name, print it elsewhere with \examname
\degree{Master of Engineering} % Your degree name, print it elsewhere with \degreename
\author{Shuo \textsc{Hou}} % Your name, print it elsewhere with \authorname
\addresses{} % Your address, print it elsewhere with \addressname

\subject{Computing & Software} % Your subject area, print it elsewhere with \subjectname
\keywords{} % Keywords for your thesis, print it elsewhere with \keywordnames
\university{\href{http://www.mcmaster.ca/}{McMaster University}} % Your university's name and URL, print it elsewhere with \univname
\department{\href{https://www.eng.mcmaster.ca/cas}{C.A.S}} % Your department's name and URL, print it elsewhere with \deptname
\group{\href{http://researchgroup.university.com}{Research Group Name}} % Your research group's name and URL, print it elsewhere with \groupname
\faculty{\href{https://www.eng.mcmaster.ca}{Faculty of Engineering}} % Your faculty's name and URL, print it elsewhere with \facname

% this sets up hyperlinks
\hypersetup{pdftitle=\ttitle} % Set the PDF's title to your title
\hypersetup{pdfauthor=\authorname} % Set the PDF's author to your name
\hypersetup{pdfkeywords=\keywordnames} % Set the PDF's keywords to your keywords

\begin{document}

 \frontmatter % Use roman page numbering style (i, ii, iii, iv...) for the pre-content pages

\pagestyle{plain} % Default to the plain heading style until the thesis style is called for the body content

%----------------------------------------------------------------------------------------
%	Half Title (lay title)
%----------------------------------------------------------------------------------------
%\begin{halftitle} % could not get this environment working
%\vspace*{\fill}
\vspace{6cm}
\begin{center}
A short 60 character title % ideally, but it doesn't seem to matter
\end{center}
%\vspace*{\fill}
\pagenumbering{gobble} % leave this here, McMaster doesn't want this page numbered
%\end{halftitle}
\clearpage

%----------------------------------------------------------------------------------------
%	TITLE PAGE
%----------------------------------------------------------------------------------------
\pagenumbering{gobble}
\begin{center}

\vfill
\textsc{\Large \ttitle} \\

\vfill
By \authorname, \\%% -----> List prior degrees after comma  <----

 \vfill
{\large \textit{A Project Report Submitted to the School of Graduate Studies in the Partial Fulfillment of the Requirements for the Degree \degreename}}\\

\vfill
{\large \univname\, \copyright\, Copyright by \authorname\, \today}\\[4cm] % replace \today with the submission date

\end{center}


%----------------------------------------------------------------------------------------
%	Descriptive note numbered ii
%----------------------------------------------------------------------------------------
% Need to add below info
\newpage
\pagenumbering{roman} % leave to turn numbering back on
\setcounter{page}{2} % leave here to make this page numbered ii, a Grad School requirement

\noindent % stops indent on next line
\univname \\ 
\degreename\, (\the\year) \\
Hamilton, Ontario (Department of Computing and Software) \\[1.5cm]
TITLE: \ttitle \\
AUTHOR: \authorname\,  %list previous degrees
(\univname)  \\
SUPERVISOR: \supname\, \\ 
NUMBER OF PAGES: \pageref{lastoffront}, \pageref{LastPage}  % put in iv and number

\clearpage

%----------------------------------------------------------------------------------------
%	Lay abstract number iii
%----------------------------------------------------------------------------------------
% not actually included in most theses, though requested by the GSA
% uncomment below lines if you want to include one
%\section*{Lay Abstract}
%\addchaptertocentry{Lay Abstract}
% Type it here
%\clearpage
%----------------------------------------------------------------------------------------
%	ABSTRACT PAGE
%----------------------------------------------------------------------------------------

\section*{\Huge Abstract} 
\addchaptertocentry{Abstract}
% Type your abstract here. 
This project assesses the classification performance of three deep neural networks on the breast cancer specimen dataset provided by western university. The dataset contains 89 samples. Class B has ** samples and class C has ** samples. Only these two classes have relatively sufficient number of samples. Three neural networks are trained on the selected data, a small neural network derived from kiret’s paper, two state-of-art convolution neural networks VGG and resnet. Machine learning often requires tens of thousands of samples. Due to the limitation of the dataset, results of all three networks are not ideal. results****. The neural network model and training process are parameterized in ymal format for more intuitive parameter fine tuning, also retraining when more data is available.\clearpage
%----------------------------------------------------------------------------------------
%	ACKNOWLEDGEMENTS
%----------------------------------------------------------------------------------------

\begin{acknowledgements}
\addchaptertocentry{\acknowledgementname} % Add the acknowledgments to the table of contents

The acknowledgments and the people to thank go here, don't forget to include your project adviser\ldots

\end{acknowledgements}

%----------------------------------------------------------------------------------------
%	LIST OF CONTENTS/FIGURES/TABLES PAGES
%----------------------------------------------------------------------------------------

\tableofcontents % Prints the main table of contents

\listoffigures % Prints the list of figures

\listoftables % Prints the list of tables

%----------------------------------------------------------------------------------------
%	ABBREVIATIONS
%----------------------------------------------------------------------------------------
% many theses don't use this section, as it will be declared at first use and again each chapter. Uncomment these four lines to activate if you want
%\clearpage
%\section*{\Huge Acronyms}
%\addchaptertocentry{Acronyms}
%\printacronyms[name] % name without an option stops the header

%----------------------------------------------------------------------------------------
%	DECLARATION PAGE
%----------------------------------------------------------------------------------------

\begin{declaration}
\addchaptertocentry{\authorshipname}

\noindent I, \authorname, declare that this thesis titled, \enquote{\ttitle} and the work presented in it are my own. I confirm that:

\begin{itemize} 
\item List each chapter
\item and what you have done for it
\end{itemize}
 
\end{declaration}


%%%%%%%%%%%%%%%%%%%%%%%%%%%
%%%%%%%%%%%%%%%%%%%%%%%%%%%
% optional page stuff
%----------------------------------------------------------------------------------------
% can do physical constraints and symbols pages, see the original thesis example on overleaf if you want to include them at https://www.overleaf.com/latex/templates/template-for-a-masters-slash-doctoral-thesis/mkzrzktcbzfl#.VlPeicorpE4
%----------------------------------------------------------------------------------------

%----------------------------------------------------------------------------------------
%	QUOTATION PAGE
%----------------------------------------------------------------------------------------

%\vspace*{0.2\textheight}

%\noindent\enquote{\itshape Thanks to my solid academic training, today I can write hundreds of words on virtually any topic without possessing a shred of information, which is how I got a good job in journalism.}\bigbreak

%\hfill Dave Barry

%----------------------------------------------------------------------------------------
%	DEDICATION
%----------------------------------------------------------------------------------------

% \dedicatory{For/Dedicated to/To my\ldots} 

%%%%%%%%%%%%%%%%%%%%%%%%%%%
%%%%%%%%%%%%%%%%%%%%%%%%%%%
%%%%%%%%%%%%%%%%%%%%%%%%%%%



%----------------------------------------------------------------------------------------
% The following bit is just here to make sure we end up on a new page and get the total number of roman numeral
\label{lastoffront}
\clearpage
% make sure this command is on the last of your frontmatter pages, i.e. only this command, a \clearpage then \mainmatter
% should be fine without modification
%----------------------------------------------------------------------------------------

%----------------------------------------------------------------------------------------
%	THESIS CONTENT - CHAPTERS
%----------------------------------------------------------------------------------------

\mainmatter % Begin numeric (1,2,3...) page numbering

\pagestyle{thesis} % Return the page headers back to the "thesis" style

% Include the chapters of the thesis as separate files from the Chapters folder
\chapter{Introduction} % Main chapter title

Breast cancer is one of the most common cancer to affect women. About 25\% among all cancer diagnosed in women is breast cancer \citep{Siegel2015}. The complex structure of tissue and highly variable image produced by Ultrasound (US) and Photoacoustic tomography (PAT) make cancer classification a challenging problem. With the recent development of machine learning algorithms, we can make use of the large amounts of image we have to help clinicians to make more accurate diagnoses. 

Generally speaking, Machine learning (ML) is a system that can automatically produce the desired output from a given input. ML, as the name suggests, could learn and improve its decision making from the many correct input-output examples that its given in the training phase. Unlike classic algorithms and programs, ML does not require manually program the desired response for all possible inputs. ML benefits from large amount of data. It could extract features and recognize general patterns from highly variant data then find the non-linear mapping from input to the correct output. 

Support Vector Machine (SVM) is a well-known classification algorithm. In many previous studies related to clinical image diagnostic, SVM has been trained to produce satisfactory results\citep{Vassis2015}. However, SVM relies on carefully hand-crafted features. One with no domain knowledge of the original data could not generalize and reuse the model on new data.

Deep Neural Network (DNN)\citep{LeCun2015}, on the other hand, is a newly developed approach that works on raw data. There is no need to precompute and extract features that based on prior knowledge about the data. Feeding a DNN model with minimally pre-processed data would be sufficient enough for it to learn then produce reliable results. DNN are extremely capable of modelling highly complex data that contain hierarchical information. For instance, a DNN could be trained to recognize faces. It does so by learning a hierarchy of visual features that distinguish faces from other objects. Starting from lower order features such as edges and corners, to features resembling parts of faces such as eyes and mouth, then finally features that are recognizable as template faces.

Convolutional Neural Network (CNN)\citep{Krizhevsky2017} is an advanced architecture of DNN. Its convolution layers allow CNN to learn models that have scale and translational invariance. Convolution layer is a CNN layer that convolves filters across an image in order to localize and synthesize important information. In a CNN, there are pooling layers work together with convolution layers. Pooling layer is to subsample images by pooling information from patches of pixels between convolutional layers. It can effectively build a hierarchy of features progressing from more local to more global in scope. This unique feature of CNN makes it well-suited for complex image classification tasks.

The primary objective of this project is to classify breast cancer specimen using CNN. Three CNN models with state-of-the-art architecture are trained on PAT and US images separately. Their classification performance is tested via k-fold cross validation. This project also parameterized the models in YAML format in order to aid parameter fine tuning.





      
% remember to set these at the start of each chapter
\chapter{Background} 
\label{background} 
This chapter discuses some important concepts in Machine Learning, the fundamentals of a Neural Network (Section \ref{backgournd_nn}), Convolutional Neural Network (Section \ref{backgournd_cnn}), and techniques in training (Section \ref{backgournd_aug}) and validating (Section \ref{backgournd_kfold}) a neural network.


\section{Neural Networks}
\label{backgournd_nn}

The term Neural Network (NN) refers to a computational model that is artificially built in computers. The model is inspired by the way biological neural networks in the human brain process information. A Neural Network is the most powerful Machine Learning model. Neural Networks are recognized for many breakthrough achievements in speech recognition, computer vision, and text processing. In this section, we will discuss the fundamentals of a Neural Network.

\subsection{A Single Neuron}
Neuron, or node, is a basic computation unit in an NN. It simply takes an input vector $\mathbf{x} = (x_0,...,x_i)$ and computes an output. Each input has an associated weight $(w_0,...,w_i)$, which is assigned on the basis of its relative importance to other inputs. The node applies a function $f$ to the weighted sum of its inputs (and a bias term $b$). The output of a node is $f(w_1x_1 +...+ w_ix_i + b)$, shown in Fig.\,\ref{node}.
\begin{figure}
	\centering
	\includegraphics[scale=0.5]{Figs/node.png}
    \caption{A node \citep{cs231n}}
    \label{node}
\end{figure}

The function $f$ is non-linear and is called the activation function. The purpose of the activation function is to introduce non-linearity into the output of a node. The following activation functions (Fig.\,\ref{activation}) are often used:
\begin{figure}
	\centering
	\includegraphics[scale=0.5]{Figs/activation.png}
    \caption{Activation functions}
    \label{activation}
\end{figure}

\begin{itemize}
  \item Sigmoid: takes a real-valued input and squashes it to range [0, 1]
        \begin{equation}
        \label{eq:sigmoid}
        \sigma(x) =  \frac{\mathrm{1} }{\mathrm{1} + e^{-x} }
        \end{equation}
  \item tanh: takes a real-valued input and squashes it to range [-1, 1]
  \begin{equation}
        \label{eq:tanh}
        \tanh(x) = 2\cdot\sigma(2x)-1 
        \end{equation}{}
  \item Softmax \citep{Goodfellow-et-al-2016}: takes in logits, outputs probabilities in range [0, 1]. All probabilities sum to 1. 
        \begin{equation}
        \label{eq:softmax}
        S(x_i) = \frac{e^{x_i}}{\sum_j e^{x_j}}
        \end{equation}
\item ReLU (Rectified Linear Unit) \citep{Nair:2010:RLU:3104322.3104425}: takes a real-valued input and thresholds it at zero
        \begin{equation}
        \label{eq:relu}
            f(x) = \max(0,x)
        \end{equation}
\end{itemize}

\subsection{Feed-forward Neural Network}

The simplest Neural Network is a feed-forward fully-connected network, also known as MultiLayer Perceptron (MLP) \citep{Orbach1962}. It is formed by three layers of nodes, one input layer, one hidden layer, and one output layer (Fig.\,\ref{one_layer}). Nodes in adjacent layers have connections between them. A connection represents a weight $w$. The capacity of a network increases with more hidden nodes and more hidden layers, neural networks with more neurons can express more complicated functions (Fig.\,\ref{anyfunction}).

\begin{figure}[h]
	\centering
	\includegraphics[scale=0.5]{Figs/1hidden.png}
    \caption{Feed-forward Neural Network  \citep{cs231n}}
    \label{one_layer}
\end{figure}

\begin{figure}[h]
	\centering
	\includegraphics[scale=0.5]{Figs/anyfunction.png}
    \caption{Non-linear decision boundary \citep{cs231n}}
    \label{anyfunction}
\end{figure}

In a feed-forward network, the information moves in only the forward direction, from the input nodes, through the hidden nodes and to the output nodes. There are no cycles in the network.

Fig.\,\ref{one_layer} is an example of feed-forward network with a single hidden layer. Each connection has a weight associated with it.

\textbf{Input layer} has three nodes. The nodes take in the input $\mathbf{x} = (x_1,x_2,x_3)$. These values are passed to the hidden layer with no computation.

\textbf{Hidden layer} has four nodes. To distinguish from the output layer, here let $v$ denote a weight and $h$ denote the output of a node in hidden layer. $D$ denotes the number of inputs, in this case $D=3$. The output of the $j$th node is $h_j(\mathbf{x}) = f(v_{j0}+\sum_{i=1}^{D=3} x_iv_{ji})$. $v_{ji}$ is a weight of the $j$th node associated with the $i$th node in input layer. $v_{j0}$ is the bias. $f$ is an activation function.

\textbf{Output Layer} has two nodes which take inputs from the hidden layer and perform similar computations.
Let $w$ denote a weight and $o$ denote the output of a node in output layer. The output of the $k$th node is $o_k(\mathbf{x}) = g(w_{k0}+\sum_{j=1}^{J=4} h_j(\mathbf{x}) w_{kj})$. $g$ is also an activation function.

Given a set of features $\mathbf{x} = (x_1,...,x_n)$ and a target $\mathbf{t} = (t_1,..,t_n)$, a neural network can learn the relationship between the features and the target, for either classification or regression.

\newcommand{\argminE}{\mathop{\mathrm{argmin}}}  

\subsection{Forward and Backward Propagation}
\label{back-prob}
There are two phases in the learning process, forward and backward. The forward phase is making output prediction from the given input. The backward phase does a backward propagation of errors. A neural network learns through back-propagation by calculating the error from the output $\mathbf{o}$ to the target value $\mathbf{t}$, then improving the weights in all nodes. The learning is supervised, meaning the target value $\mathbf{t}$ must be given to the network respect to each input  $\mathbf{x}$ in training. A loss function $L$ measures the error between network output $\mathbf{o}$ and the target value $\mathbf{t}$. The objective of learning is to obtain $\mathbf{w}^* = \argminE_\mathbf{w} \sum_{n=1}^N L(\mathbf{o}^{(n)},\mathbf{t}^{(n)})$, where $\mathbf{o} = f(\mathbf{x};\mathbf{w})$ is the output of the network. With hidden nodes, this objective is not convex. We can minimize the function by gradient descent (a gradient method may not end up in a local minima/saddle point, but experimental evidence shows that most local minimas can be almost as good as the global minima).

 Let $E$ denote the error measured by $L$. For clarity, we can expand the output layer as shown in Fig.\,\ref{expand2}. $o_k$ is the output of unit $k$, $g$ is the output layer activation function, $z_{k}$ is the net input to output unit $k$, $w_{ki}$ is the weight from input $i$ to $k$. 
\begin{figure}[h]
	\centering
	\includegraphics[scale=0.5]{Figs/singlelayer2.png}
    \caption{Single layer network}
    \label{expand2}
\end{figure}
Error gradients for single layer network is 
$$\frac{\partial E}{\partial w_{ki}} = 
    \frac{\partial E}{\partial o_{k}} 
    \frac{\partial o_{k}}{\partial z_{k}} 
    \frac{\partial z_{k}}{\partial w_{ki}} $$
Error at output $o_k$ is $\delta_k^o =  \frac{\partial E}{\partial o_{k}} $.
$$\frac{\partial E}{\partial w_{ki}} = 
    \frac{\partial E}{\partial o_{k}} 
    \frac{\partial o_{k}}{\partial z_{k}} 
    \frac{\partial z_{k}}{\partial w_{ki}} 
    =  
    \delta_k^o 
    \frac{\partial o_{k}}{\partial z_{k}} 
    \frac{\partial z_{k}}{\partial w_{ki}}  $$
Since $o_k = g(z_k)$, $$\delta_k^z = \delta_k^o \cdot \frac{\partial o_{k}}{\partial z_{k}}$$ then we have $$\frac{\partial E}{\partial w_{ki}} = \delta_k^z \frac{\partial z_{k}}{\partial w_{ki}} =  \delta_k^z \cdot x_i$$ 
Assuming the loss function is mean-squared error (MSE), on a single training example $n$, we have:
$$\frac{\partial E}{\partial o_k^{(n)}} = o_k^{(n)} - t_k^{(n)} := \delta_k^o$$Using logistic activation functions:$$ o_k^{(n)} = g(z_k^{(n)}) = (1 + \exp(-z_k^{(n)}))^{-1}$$
$$\frac{\partial o_k^{(n)}}{\partial z_k^{(n)}} = o_k^{(n)}(1-o_k^{(n)})$$
The error gradient is then:
$$\frac{\partial E}{\partial w_{ki}} =  \sum_{n=1}^N \frac{\partial E}{\partial o_{k}^{(n)}}\frac{\partial o_{k}^{(n)}}{\partial z_{k}^{(n)}} \frac{\partial z_{k}^{(n)}}{\partial w_{ki}} = \sum_{n=1}^N (o_k^{(n)} - t_k^{(n)})o_k^{(n)} (1-o_k^{(n)}) x_i^{(n)}$$
The gradient descent update rule is given by:
$$w_{ki} \leftarrow w_{ki} - \eta \frac{\partial E}{\partial w_{ki}} = w_{ki} - \eta  \sum_{n=1}^N (o_k^{(n)} - t_k^{(n)})o_k^{(n)} (1-o_k^{(n)}) x_i^{(n)}$$
\begin{figure}[h]
	\centering
	\includegraphics[scale=0.5]{Figs/multilayerbackprop.png}
    \caption{Multi layer network}
    \label{multi}
\end{figure}
where $\eta$ is a pre-defined learning rate. The output weight gradients for a multi-layer network are the same as for a single layer network (Fig.\,\ref{multi}):
$$\frac{\partial E}{\partial w_{kj}} =  \sum_{n=1}^N \frac{\partial E}{\partial o_{k}^{(n)}}\frac{\partial o_{k}^{(n)}}{\partial z_{k}^{(n)}} \frac{\partial z_{k}^{(n)}}{\partial w_{kj}} = \sum_{n=1}^N \delta_k^{z,(n)} h_j^{(n)}$$
where $\delta_k$ is the error w.r.t. the net input for unit $k$. Hidden weight gradients are then computed via back propagation:
$$\frac{\partial E}{\partial h_{j}^{(n)}} =  \sum_{k} \frac{\partial E}{\partial o_{k}^{(n)}}\frac{\partial o_{k}^{(n)}}{\partial z_{k}^{(n)}} \frac{\partial z_{k}^{(n)}}{\partial h_{j}^{(n)}} = \sum_{k} \delta_k^{z,(n)} w_{kj} := \delta_j^{h,(n)}$$
$$ \frac{\partial E}{\partial v_{ji}} 
= \sum_{n=1}^N \frac{\partial E}{\partial h_{j}^{(n)}}\frac{\partial h_{j}^{(n)}}{\partial u_{j}^{(n)}} \frac{\partial u_j^{(n)}}{\partial v_{ji}}
= \sum_{n=1}^N \delta_j^{h,(n)} f'(u_j^{(n)})\frac{\partial u_j^{(n)}}{\partial v_{ji}} 
= \sum_{n=1}^N \delta_j^{u,(n)} x_i^{(n)} $$

Then, a gradient descent update is performed on hidden weights. $\eta$ is the learning rate.
$$v_{ji} \leftarrow v_{ji} - \eta \frac{\partial E}{\partial v_{ji}}$$

Often times, the learning rate $\eta$ is difficult to set. If it is too large, the local and global minima can be “overshot”, leading to slow convergence, and if the learning rate is too small, it can take a long time to converge to an acceptable minimum. Adaptive Moment Estimation (Adam) \citep{adam} optimizer was designed to adaptively set the learning rate during training.

\begin{figure}[h]
	\centering
	\includegraphics[scale=0.4]{Figs/dropout.png}
    \caption{Neural network with dropout \citep{JMLR:v15:srivastava14a}}
    \label{dropout}
\end{figure}

Neural networks have the ability to fit very complicated functions. Overfitting may occur when a neural network fits not only valid data but also noise. Dropout \citep{JMLR:v15:srivastava14a} is a technique to prevent a neural network from overfitting. Dropout is to temporarily remove (with a dropout probability) a node from the network, along with all its incoming and outgoing connections (Fig.\,\ref{dropout}).


\section{Convolutional Neural Networks}
The architecture of a Convolutional Neural Network (CNN) was inspired by the organization of the Visual Cortex in the human brain \citep{Fukushima2007}. Individual neurons respond to stimuli only in a restricted region of the visual field known as the Receptive Field. A collection of such fields overlap to cover the entire visual area. Similarly, a CNN model takes in an input image, assign importance (weights and biases) to regions of pixels via kernels/filters. Kernels convolve across the whole image. These kernels are learnable in the training process to extract different features of the image.

\subsection{Convolution Layer — The Kernel}
\label{backgournd_cnn}

Fig.\,\ref{kernels} shows four line detection kernels which respond maximally to horizontal, vertical and oblique (+45 and - 45 degree) single pixel wide lines.

\begin{figure}[h]
	\centering
	\includegraphics[scale=0.6]{Figs/kernels.png}
    \caption{Kernels}
    \label{kernels}
\end{figure}

Convolution is done by matrix multiplication operation between the kernel and a portion of the image over which the kernel is hovering. The kernel is moved from left to right, top to bottom with a certain stride value. Fig.\,\ref{convlayer} shows a step of kernel application on image. 

\begin{figure}[h]
	\centering
	\includegraphics[scale=0.4]{Figs/convlayer.png}
    \caption{Applying a kernel on image \citep{towarddata}}
    \label{convlayer}
\end{figure}


CNN is not limited to only one convolutional layer. Conventionally, the first convolutional layer is responsible for capturing low-level features such as edges, color, gradient orientation, etc. With added layers, the architecture adapts to high-level features as well, giving us a network which has the wholesome understanding of images in the dataset.

\subsection{Pooling Layer}

Similar to the convolutional layer, the pooling layer is responsible for reducing the spatial size of the convolved feature. This is to decrease the computational power required to process the data through dimensionality reduction. Furthermore, it is useful for extracting dominant features which are rotational and positional invariant.

There are two types of pooling, max pooling and average pooling. Max pooling returns the maximum value from the portion of the image covered by the kernel. On the other hand, average pooling returns the average of all the values from the portion of the image covered by the kernel. They perform dimensionality reduction alone with noise reduction.

Some number of convolutional layers and pooling layers together form a block of a Convolutional Neural Network. Depending on the complexities in the images, the number of such blocks may be increased for capturing low-levels details even further, but the more layers CNN has the more computation it performs.

\subsection{Classification — Fully Connected Layer}

A fully-connected layer is used to learn non-linear combinations of the high-level features as represented by the output of the convolutional layer. The fully connected layer is learning a non-linear function of the features and the correct output.

The input image is now converted into many features through convolutional layers. The output is then flattened to be fed to a feed-forward neural network and back propagation applied to every iteration of training. Over a series of epochs, the model is able to distinguish between dominating and certain low-level features in images and classify them using the Softmax activation function (\ref{eq:softmax}).

\section{K-Fold Cross-Validation}
\label{backgournd_kfold}
K-Fold Cross-Validation \citep{Kohavi95astudy} (Fig.\,\ref{kfold}) is where a given data set is split into a K sets/folds where each fold is used as a testing set at some point. For example, in an application of a 5-Fold cross-validation $K=5$, the dataset is split into 5 folds. In the first iteration, the first fold is used to test the model and the rest are used to train the model. In the second iteration, second fold is used as the testing set while the rest serve as the training set. This process is repeated until each fold of the five folds have been used as the testing set. The advantage of this method over repeated random sub-sampling is that all observations are used for both training and validation, and each observation is used for validation exactly once. The testing accuracy of each fold will be averaged to evaluate the model's performance. 

\begin{figure}[h]
	\centering
	\includegraphics[width=\textwidth]{Figs/kfold.png}
    \caption{5-Fold Cross Validation \citep{k-fold}}
    \label{kfold}
\end{figure}


\section{Image Data Augmentation}
\label{backgournd_aug}

The performance of deep learning neural networks often improves with the amount of data available. Data augmentation is a technique to artificially create new training data from existing training data \citep{Mikolajczyk2018}. This is done by applying domain-specific techniques to examples from the training data that create new and different training examples.

\begin{figure}[h]
	\centering
	\includegraphics[width=.5\textwidth]{Figs/dataaug.jpg}
    \caption{Image data augmentation}
    \label{dataaug}
\end{figure}

Image data augmentation is to create transformed versions of images in the training dataset that belong to the same class as the original image. Transforms include a range of operations from the field of image manipulation, such as shifting, flipping, zooming, cropping, rotating, and blurring. Fig.\,\ref{dataaug} shows examples of generated images by data augmentation from our dataset.

Convolutional neural network can learn features that are invariant to their location in the image. Image augmentation can aid the model in learning features that are invariant to transforms, such as left-to-right to top-to-bottom ordering. Image data augmentation is typically only applied to the training dataset, and not to the validation or test dataset. 



 
% remember to set these at the start of each chapter
\chapter{CNN Architecture} 
\label{arcgutecture} 

%%%%%%%%%%%%%%%%%%
\section{VGG}
\section{ResNet}
\section{Small CNN}

% remember to set these at the start of each chapter
\chapter{Experiments and Discussion} 
\label{experiments}
In this section, we show and discuss the performance of each CNN model, including Small CNN, VGG, VGG-IN, and ResNet. Training is performed on PAT and US datasets and two other datasets for verification. Section \ref{result_dataset} and \ref{result_aug} describe the dataset and data pre-processing. Section \ref{result_training} describes the training process, and Section \ref{result_curves} and \ref{result_acc} include the experimental results. We then have a discussion about the results in Section \ref{result_discussion}. Finally, the experimental results on other datasets are shown in Section \ref{result_other}.
%%%%%%%%%%%%%%%%%%
\section{Dataset}
\label{result_dataset}
The dataset is obtained from the study of \cite{Kosik2019}. It contains 90 breast cancer tumour samples, each of which has an Ultrasound (US) image stack and a Photoacoustic (PAT) image stack. In the dataset, each sample belongs to one of 14 types (class A - N) of cancer, where the actual cancer name is masked to obtain a blind study. The distribution of cancer classes is shown in Fig.\,\ref{class_graph}.  
\begin{figure}[h]
	\centering
	\includegraphics[width=0.7\textwidth]{Figs/class_graph.png}
    \caption{Cancer types and number of samples per type}
    \label{class_graph}
\end{figure}

B has 33 samples and C has 21 samples. In this project, only class B and C are used, because the other classes have too few samples to train a neural network.

A specimen of breast cancer tumour is scanned in the $z$ direction (vertical) relative to the holding platform. Scans produce a stack of 2D images of the cross-section. An example of an image stack is shown in Fig.\,\ref{stack}. 

\begin{figure}[h]
	\centering
	\includegraphics[width=\textwidth]{Figs/pat_stack_50.jpg}
    \caption{US stack example}
    \label{stack}
\end{figure}

We can see that the top and bottom scans, for example 1, 2, 3, 39, 40, 41 in Fig.\,\ref{stack}, are very small and lack of detail of the internal texture. This is because the cross-sections close to the top and bottom of a tumour specimen are very small. The cross-sections close to the centre of a specimen, such as 18, 19, 21, 21 are larger in size and have more texture. For each image stack, US and PAT, the centre 6 images are extracted (we could extract more or even use all of the images, but we may not want to use images with poor quality) to use for training and validation. There are in total 389 PAT images and 344 US images extracted from the stacks. Examples of extracted PAT and US images are shown in Fig.\,\ref{pat_us_example}.

\begin{figure}
\centering
\begin{subfigure}[b]{.24\linewidth}
\includegraphics[width=\linewidth]{Figs/PAT014_18.jpg}
\caption{AF014 PAT}
\end{subfigure}
\begin{subfigure}[b]{.24\linewidth}
\includegraphics[width=\linewidth]{Figs/PAT023_23.jpg}
\caption{AF023 PAT}
\end{subfigure}
\begin{subfigure}[b]{.24\linewidth}
\includegraphics[width=\linewidth]{Figs/PAT36.png}
\caption{AF036 PAT}
\end{subfigure}
\begin{subfigure}[b]{.24\linewidth}
\includegraphics[width=\linewidth]{Figs/PAT055_11.jpg}
\caption{AF055 PAT}
\end{subfigure}

\begin{subfigure}[b]{.24\linewidth}
\includegraphics[width=\linewidth]{Figs/US14_18.jpg}
\caption{AF014 US}
\end{subfigure}
\begin{subfigure}[b]{.24\linewidth}
\includegraphics[width=\linewidth]{Figs/US23_23.jpg}
\caption{AF023 US}
\end{subfigure}
\begin{subfigure}[b]{.24\linewidth}
\includegraphics[width=\linewidth]{Figs/US36.png}
\caption{AF036 US}
\end{subfigure}
\begin{subfigure}[b]{.24\linewidth}
\includegraphics[width=\linewidth]{Figs/US55_13.jpg}
\caption{AF055 US}
\end{subfigure}
\caption{Extracted PAT and US images}
\label{pat_us_example}
\end{figure}

\section{Data Partition and Augmentation}
\label{result_aug}

The images are divided into a training set and a testing set by an 80/20 ratio. No image from the same stack is separated into the training and the testing set. Because the number of samples in class B and C is not balanced, K-fold data partitioning is done using a Stratified k-fold. Stratified k-fold is where the training set and the testing set contains approximately the same percentage of samples of each target class as the original dataset. Each of the five folds is used as a test set only once. Accuracy and $F_1$ score (defined as (\ref{f1_score})) are averaged across five folds.

Machine learning generally requires large amounts of data. Since our dataset is small, data augmentation is used to artificially expand the number of samples. Augmentation Scaling Factor is set to 5, meaning five images are created from one by rotation, cropping, horizontal and vertical flip. Then, the images are resized to (192, 192).

\section{Training}
\label{result_training}

The four CNN models, Small CNN, VGG, VGG-IN, and ResNet are implemented and trained using the open-source neural-network library Keras \citep{chollet2015keras} in Python \citep{van1995python}. Adam \citep{adam} optimizer is used in all models to minimize the loss function. Adam can adaptively adjust the learning parameters based on the average first moment, as well as the average of the second moments of the gradients. The parameters of Adam optimizer are set to their default. Number of epoch is 50, as all samples are passed to the network 50 times, with a batch size of 32. Checkpoints are used so that the best set of weights is saved. For a classification problem, the loss function is categorical cross-entropy loss \citep{Goodfellow-et-al-2016}, which is defined as $$-\sum^M_{c=1} y_{o,c}\log(p_{o,c})$$
$M$ is the number of classes, $y$ is the binary indicator (0 or 1) if class label $c$ is the correct classification for observation $o$, and $p$ is the predicted probability observation $o$ is of class $c$. 

Training a neural network is a ccomputationally intensive task. To speed up the training process, the program is run on a supercomputer system SHARCNET \citep{sharcnet}.

\section{Accuracy and $F_1$ score}
\label{result_acc}

The $F_1$ score (also called F-score) is a measure of a test's accuracy \citep{powers2011evaluation}. It considers both the precision $p$ and the recall $r$ of the test to compute the score. The $F_1$ score is the harmonic average of the precision and recall, where an $F_1$ score reaches its best value at 1 (perfect precision and recall) and worst at 0.
Formally: 
\begin{equation}\label{f1_score}
    F_1 = 2 \cdot \frac{\mathrm{precision} \cdot \mathrm{recall}}{\mathrm{precision} + \mathrm{recall}} = 2 \cdot \frac{p \cdot r}{p+r}
\end{equation}
where precition is: $$p = \frac{TP}{TP + FP}$$
and recall is: $$r = \frac{TP}{TP + FN}$$
\noindent Here TP denotes the number of true positives: outcomes where the model correctly predicts the positive class.

\noindent TN denotes the number of true negatives: outcomes where the model correctly predicts the negative class.

\noindent FP denotes the number of false positives: outcomes where the model incorrectly predicts the positive class.

\noindent FN denotes the number of false negatives: outcomes where the model incorrectly predicts the negative class.

\begin{table}[h]
\centering
\begin{tabular}{ |p{4cm}||p{3cm}|p{3cm}|p{3cm}|  }
 \hline
 Model       & Accuracy & Class B $F_1$ score & Class C $F_1$ score\\
 \hline
 \hline
 Small  US   & 0.66  & 0.77 &  0.23\\
 ResNet US   & \textbf{0.75}  & \textbf{0.82} &  \textbf{0.55}\\
 VGG US      & 0.61  & 0.76 &  0\\
 VGG-IN US & 0.65 & 0.77 & 0.22 \\
\hline
 Small PAT   & 0.71  & 0.79 &  0.50\\
 ResNet PAT  & 0.76  & 0.83 &  \textbf{0.57}\\
 VGG PAT     & 0.61  & 0.76 &  0\\
 VGG-IN PAT & \textbf{0.78} & \textbf{0.84} & 0.54 \\
 \hline
\end{tabular}
\caption{Model accuracy and $F_1$ score on the US and PAT datasets}
\label{acctable}
\end{table}

\section{Training Curves}
\label{result_curves}
\label{section_curves}

\begin{figure}
\centering
\begin{subfigure}[b]{.45\linewidth}
\includegraphics[width=\linewidth]{Figs/small_us_loss.jpg}
\caption{Small CNN US}
\end{subfigure}
\begin{subfigure}[b]{.45\linewidth}
\includegraphics[width=\linewidth]{Figs/small_pat_loss.jpg}
\caption{Small CNN PAT}
\end{subfigure}

\begin{subfigure}[b]{.45\linewidth}
\includegraphics[width=\linewidth]{Figs/resnet_us_loss.jpg}
\caption{ResNet US}
\end{subfigure}
\begin{subfigure}[b]{.45\linewidth}
\includegraphics[width=\linewidth]{Figs/resnet_pat_loss.jpg}
\caption{ResNet PAT}
\end{subfigure}

\begin{subfigure}[b]{.45\linewidth}
\includegraphics[width=\linewidth]{Figs/vgg_us_loss.jpg}
\caption{VGG US}
\end{subfigure}
\begin{subfigure}[b]{.45\linewidth}
\includegraphics[width=\linewidth]{Figs/vgg_pat_loss.jpg}
\caption{VGG PAT}
\end{subfigure}

\begin{subfigure}[b]{.45\linewidth}
\includegraphics[width=\linewidth]{Figs/vgg_in_us_loss.jpg}
\caption{VGG-IN US}
\end{subfigure}
\begin{subfigure}[b]{.45\linewidth}
\includegraphics[width=\linewidth]{Figs/vgg_in_pat_loss.jpg}
\caption{VGG-IN PAT}
\end{subfigure}
\caption{Model loss on the US and PAT datasets}
\label{fig:loss}
\end{figure}

\begin{figure}
\centering
\begin{subfigure}[b]{.45\linewidth}
\includegraphics[width=\linewidth]{Figs/small_us_acc.jpg}
\caption{Small CNN US}
\end{subfigure}
\begin{subfigure}[b]{.45\linewidth}
\includegraphics[width=\linewidth]{Figs/small_pat_acc.jpg}
\caption{Small CNN PAT}
\end{subfigure}

\begin{subfigure}[b]{.45\linewidth}
\includegraphics[width=\linewidth]{Figs/resnet_us_acc.jpg}
\caption{ResNet US}
\end{subfigure}
\begin{subfigure}[b]{.45\linewidth}
\includegraphics[width=\linewidth]{Figs/resnet_pat_acc.jpg}
\caption{ResNet PAT}
\end{subfigure}

\begin{subfigure}[b]{.45\linewidth}
\includegraphics[width=\linewidth]{Figs/vgg_us_acc.jpg}
\caption{VGG US}
\end{subfigure}
\begin{subfigure}[b]{.45\linewidth}
\includegraphics[width=\linewidth]{Figs/vgg_pat_acc.jpg}
\caption{VGG PAT}
\end{subfigure}

\begin{subfigure}[b]{.45\linewidth}
\includegraphics[width=\linewidth]{Figs/vgg_in_us_acc.jpg}
\caption{VGG-IN US}
\end{subfigure}
\begin{subfigure}[b]{.45\linewidth}
\includegraphics[width=\linewidth]{Figs/vgg_in_pat_acc.jpg}
\caption{VGG-IN PAT}
\end{subfigure}
\caption{Model accuracy on the US and PAT datasets}
\label{fig:acc}
\end{figure}

The learning process of a neural network can be investigated through training curves \citep{Anzanello2011}. Training and validation loss and accuracy are reported and ploted in Fig.\,\ref{fig:loss} and Fig.\,\ref{fig:acc}. 


\section{Discussion}
\label{result_discussion}

For the Small CNN model, even though the training loss is decreasing, the validation loss is increasing (Fig.\,\ref{fig:loss} (A) (B)). This model is not generalizing to fit unseen validation data. For ResNet model, the decrease of training and validation loss (Fig.\,\ref{fig:loss} (C) (D)) indicates that ResNet could improve from training (although very noisy). VGG model has almost the same depth as ResNet, 19 layers versus 18 layers. Yet, the convergence of training and validation loss is very poor. We do not see a decrease in training and validation loss (Fig.\,\ref{fig:loss} (E) (F)). VGG-IN model uses a set of pre-trained weights from ImageNet \citep{imagenet_cvpr09} in convolutional layers, only the fully-connected layers and classification layer are trained. VGG-IN model shows decreases in training and validation loss on US dataset (Fig.\,\ref{fig:loss}), but on PAT dataset (Fig.\,\ref{fig:loss} (H)) only training loss decreases.

All models perform poorly in terms of accuracy on the US and PAT datasets. The validation accuracy of all models do not show a consistent improvement over epoch (Fig.\,\ref{fig:acc}). Note the large gap between training and validation curves. This indicates that the training dataset is unrepresentative, which means the training dataset does not provide sufficient information to learn the problem. We may have too few examples in the training dataset, or the training dataset may not be a good representation. From the noisy validation curves, we can conclude that the validation dataset is also unrepresentative. A unrepresentative validation set means that it does not provide sufficient information to evaluate the ability of the model to generalize. This could occur when the validation dataset has too few examples.

Medical images are naturally difficult to acquire. By machine learning standard, which often requires thousands or even millions of samples, our datasets is considered extremely small. In addition, the US and PAT scans are very noisy and have lots of artifacts. The low quality of images makes it more challenging for neural networks to pick up important features. With additional data come to hand, we hope to see an improvement on the models' performance.

\section{Experiments with other datasets}
\label{result_other}

\begin{figure}[h]
\centering
\begin{subfigure}[b]{.2\linewidth}
\includegraphics[width=\linewidth]{Figs/8864_idx5_x1251_y1651_class0.png}
\end{subfigure}
\begin{subfigure}[b]{.2\linewidth}
\includegraphics[width=\linewidth]{Figs/8864_idx5_x1201_y1801_class0.png}
\end{subfigure}
\begin{subfigure}[b]{.2\linewidth}
\includegraphics[width=\linewidth]{Figs/8864_idx5_x1201_y1701_class0.png}
\end{subfigure}
\begin{subfigure}[b]{.2\linewidth}
\includegraphics[width=\linewidth]{Figs/8864_idx5_x1151_y1701_class0.png}
\end{subfigure}

\begin{subfigure}[b]{.2\linewidth}
\includegraphics[width=\linewidth]{Figs/8864_idx5_x1851_y2251_class1.png}
\end{subfigure}
\begin{subfigure}[b]{.2\linewidth}
\includegraphics[width=\linewidth]{Figs/8864_idx5_x1801_y2701_class1.png}
\end{subfigure}
\begin{subfigure}[b]{.2\linewidth}
\includegraphics[width=\linewidth]{Figs/8864_idx5_x1801_y2651_class1.png}
\end{subfigure}
\begin{subfigure}[b]{.2\linewidth}
\includegraphics[width=\linewidth]{Figs/8864_idx5_x1801_y2551_class1.png}
\end{subfigure}
\caption{Examples from the IDC Breast Cancer dataset}
\label{IDC}
\end{figure}

\begin{figure}[h]
\centering
\begin{subfigure}[b]{.2\linewidth}
\includegraphics[width=\linewidth]{Figs/cat450.jpg}
\end{subfigure}
\begin{subfigure}[b]{.2\linewidth}
\includegraphics[width=\linewidth]{Figs/cat1921.jpg}
\end{subfigure}
\begin{subfigure}[b]{.2\linewidth}
\includegraphics[width=\linewidth]{Figs/cat1412.jpg}
\end{subfigure}
\begin{subfigure}[b]{.2\linewidth}
\includegraphics[width=\linewidth]{Figs/cat1168.jpg}
\end{subfigure}

\begin{subfigure}[b]{.2\linewidth}
\includegraphics[width=\linewidth]{Figs/dog876.jpg}
\end{subfigure}
\begin{subfigure}[b]{.2\linewidth}
\includegraphics[width=\linewidth]{Figs/dog508.jpg}
\end{subfigure}
\begin{subfigure}[b]{.2\linewidth}
\includegraphics[width=\linewidth]{Figs/dog4133.jpg}
\end{subfigure}
\begin{subfigure}[b]{.2\linewidth}
\includegraphics[width=\linewidth]{Figs/dog1178.jpg}
\end{subfigure}
\caption{Examples from the Cats vs. Dogs dataset}
\label{catdog}
\end{figure}

Due to the poor performance in experiments on our US and PAT datasets, to verify the capabilities of these neural network models, they are trained on two other classification datasets, the IDC Breast Cancer dataset \citep{Janowczyk2016} (Fig.\,\ref{IDC}) and the Dogs vs. Cats dataset \citep{catdog}  (Fig.\,\ref{catdog}).


Table \ref{stats} shows the number of images in the datasets. The models are trained for 10 epoch on the IDC Breast Cancer Dataset and 30 epoch on the Dogs vs. Cats Dataset.
\begin{table}[h]
\centering
\begin{tabular}{ |p{3cm}||p{3cm}|p{3cm}|  }
 \hline
 Class       & Training set & Validation set\\
 \hline
 \hline
 IDC class 0   & 179,343   &  68,549 \\
 IDC class 1  & 71,609  & 27,763\\
 \hline
 Cat   & 10,499  &  2,003\\
 Dog  & 10,499  &  2,003\\
 \hline
\end{tabular}
\caption{Number of images in each set}
\label{stats}
\end{table}


\begin{table}[h]
\centering
\begin{tabular}{ |p{4cm}||p{3cm}|p{3cm}|p{3cm}|  }
 \hline
 Model       & Accuracy & Class 0 $F_1$ score & Class 1 $F_1$ score\\
 \hline
 \hline
 Small Breast   & \textbf{0.89}  & \textbf{0.92} &  \textbf{0.80}\\
 ResNet Breast  & 0.88  & \textbf{0.92} &  \textbf{0.80}\\
 VGG Breast      & 0.71  & 0.83 &  0\\
 VGG-IN breast & 0.85 & 0.90 & 0.73 \\
 \hline
 Small CatDog   & \textbf{0.93}  & \textbf{0.93} &  \textbf{0.93}\\
 ResNet CatDog  & \textbf{0.93}  & \textbf{0.93} &  \textbf{0.93}\\
 VGG CatDog      & 0.50  & 0.67 &  0\\
 VGG-IN CatDog  & 0.92 & 0.92 & 0.92 \\
  \hline
\end{tabular}
\caption{Model accuracy and $F_1$ score on the IDC Breast Cancer dataset and the Dogs vs. Cats dataset}
\label{acctable2}
\end{table}

Validation accuracy and $F_1$ score are reported in Table \ref{acctable2}. The Small CNN model and the ResNet model achieved the same accuracy and $F_1$ score on both datasets. the VGG model is not training well. Using pre-trained weights in the VGG-IN model is effective to improve the VGG model, but the performance still does not match the Small CNN model and the ResNet model. A deep neural network model such as VGG is very difficult to train due to the degradation problem. A deep network not necessarily performs better than a shallower network.

Training loss and accuracy plots are shown in Fig.\,\ref{fig:loss2}. 

\begin{figure}[h]
\centering
\begin{subfigure}[b]{.45\linewidth}
\includegraphics[width=\linewidth]{Figs/small_breast_loss.jpg}
\caption{Small Breast}
\end{subfigure}
\begin{subfigure}[b]{.45\linewidth}
\includegraphics[width=\linewidth]{Figs/small_catdog_loss.jpg}
\caption{Small CatDog}
\end{subfigure}

\begin{subfigure}[b]{.45\linewidth}
\includegraphics[width=\linewidth]{Figs/resnet_breast_loss.jpg}
\caption{ResNet Breast}
\end{subfigure}
\begin{subfigure}[b]{.45\linewidth}
\includegraphics[width=\linewidth]{Figs/resnet_catdog_loss.jpg}
\caption{ResNet CatDog}
\end{subfigure}

\begin{subfigure}[b]{.45\linewidth}
\includegraphics[width=\linewidth]{Figs/vgg_breast_loss.jpg}
\caption{VGG Breast}
\end{subfigure}
\begin{subfigure}[b]{.45\linewidth}
\includegraphics[width=\linewidth]{Figs/vgg_catdog_loss.jpg}
\caption{VGG CatDog}
\end{subfigure}

\begin{subfigure}[b]{.45\linewidth}
\includegraphics[width=\linewidth]{Figs/vgg_in_breast_loss.jpg}
\caption{VGG-IN Breast}
\end{subfigure}
\begin{subfigure}[b]{.45\linewidth}
\includegraphics[width=\linewidth]{Figs/vgg_in_catdog_loss.jpg}
\caption{VGG-IN CatDog}
\end{subfigure}

\caption{Model loss on the IDC Breast Cancer dataset and the Dogs vs. Cats dataset}
\label{fig:loss2}
\end{figure}

\begin{figure}
\centering
\begin{subfigure}[b]{.45\linewidth}
\includegraphics[width=\linewidth]{Figs/small_breast_acc.jpg}
\caption{Small Breast}
\end{subfigure}
\begin{subfigure}[b]{.45\linewidth}
\includegraphics[width=\linewidth]{Figs/small_catdog_acc.jpg}
\caption{Small CatDog}
\end{subfigure}

\begin{subfigure}[b]{.45\linewidth}
\includegraphics[width=\linewidth]{Figs/resnet_breast_acc.jpg}
\caption{ResNet Breast}
\end{subfigure}
\begin{subfigure}[b]{.45\linewidth}
\includegraphics[width=\linewidth]{Figs/resnet_catdog_acc.jpg}
\caption{ResNet CatDog}
\end{subfigure}

\begin{subfigure}[b]{.45\linewidth}
\includegraphics[width=\linewidth]{Figs/vgg_breast_acc.jpg}
\caption{VGG Breast}
\end{subfigure}
\begin{subfigure}[b]{.45\linewidth}
\includegraphics[width=\linewidth]{Figs/vgg_catdog_acc.jpg}
\caption{VGG CatDog}
\end{subfigure}

\begin{subfigure}[b]{.45\linewidth}
\includegraphics[width=\linewidth]{Figs/vgg_in_breast_acc.jpg}
\caption{VGG-IN Breast}
\end{subfigure}
\begin{subfigure}[b]{.45\linewidth}
\includegraphics[width=\linewidth]{Figs/vgg_in_catdog_acc.jpg}
\caption{VGG-IN CatDog}
\end{subfigure}

\caption{Model accuracy on the IDC Breast Cancer dataset and the Dogs vs. Cats dataset}
\label{fig:acc2}
\end{figure}


% I suggest only compiling one chapter at a time, and comment out the others. That way, the document will typeset faster. When your done with all the chapters, then uncomment them all. Don't worry about the numbering of chapters/figures/etc. LaTeX will take care of that. 

%----------------------------------------------------------------------------------------
%	THESIS CONTENT - APPENDICES
%----------------------------------------------------------------------------------------

\appendix % Cue to tell LaTeX that the following "chapters" are Appendices
\renewcommand{\thetable}{A\arabic{chapter}.\arabic{table}} % adds an A to table names in appendix (Table A1.1, A1.2...)
\renewcommand{\thefigure}{A\arabic{chapter}.\arabic{figure}} % same for figures
\renewcommand{\thesection}{A\arabic{section}}

% Include the appendices of the thesis as separate files from the Appendices folder
\chapter{Chapter 1 Supplement} % Main appendix title

\label{Supp_chap1}

Here is the supplemental file for chapter 1, as an alternative to including it in the main \TeX for chapter 1. This allows the figures/table to be numbered in a different way and for the section to be listed in the table of contents differently. 


\begin{figure}[h] % put figure roughly here, will float though
	\centering
	\includegraphics[scale=0.4]{Figs/Wright_1932_1.pdf}
    \caption[Same Fig]{Same fig as before, but in the appendix!}
    \label{Another_fig}
\end{figure}


%----------------------------------------------------------------------------------------
%	BIBLIOGRAPHY
%----------------------------------------------------------------------------------------

\printbibliography[heading=bibintoc]

%----------------------------------------------------------------------------------------

\end{document}