\chapter{Instruction For Running the Model} % Main appendix title
\label{appendixA}

Required packages: keras, sklearn, pandas, numpy, PIL, pyyaml.

split\_stack.py: The initial PAT and US images that we received were in a 3-dimensional image stack (.tif) format. This script is to quickly process these stacks and generate a corresponding set of 2D images.
The original dataset has the following directory structure:

\dirtree{%
.1 dataset.
.2 AF014.
.3 PAT 930.
.4 014 PAT 930 initial stack.tif .
.3 US.
.4 14.tif .
.2 AF018.
.2 ..
.2 ..
}
Output directory structure produced by the script:
\dirtree{%
.1 PAT\_extracted.
.2 AF014.
.3 014\_15.jpg .
.3 014\_16.jpg .
.3 014\_17.jpg .
.3 014\_18.jpg .
.3 014\_19.jpg .
.3 014\_20.jpg .
.2 AF018.
.2 ..
.2 ..
}

data.py partitions the data into k-fold, prepares the directory as needed by flow\_from\_directory() for training (this step will be done automatically in training). Directory structure is:
\dirtree{%
.1 data .
.2 train.
.3 B .
.4 014\_15.jpg .
.4 014\_16.jpg .
.4 ..
.3 C .
.4 027\_14.jpg .
.4 027\_15.jpg .
.4 ..
.2 valid.
.3 B .
.3 C .
}

trin.py: Main function, includes reading training set, k-fold partitioning, training model, and reporting training and validation loss and accuracy. A configuration YAML file should be provided to train.py. Path to dataset, model used, and many training parameters are specified in the file.

To run the model, first:

python split\_stack.py train

Second:

python train.py train
