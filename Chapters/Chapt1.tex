\chapter{Introduction} % Main chapter title

Breast cancer is one of the most common cancer to affect women. About 25\% among all cancers diagnosed in women is breast cancer \citep{Siegel2015}. The complex structure of tissue and highly variable images produced by Ultrasound (US) and Photoacoustic tomography (PAT) make cancer classification a challenging problem. With the recent development of machine learning algorithms, we can make use of the large amounts of images to help clinicians to make more accurate diagnoses. 

\citeauthor{Kosik2019} have developed a method to produce image stacks from US and PAT scans of breast cancer tumour \citep{Kosik2019}. They provided us with a dataset containing US and PAT image stacks of 88 cancer tumours and their tumour types. The primary objective of this project is to classify breast cancer tumours using Deep Learning. Four state-of-the-art Deep Neural Network models are trained on the dataset. Their classification performance is compared. This project also parameterizes the training of these models in a YAML \citep{yaml} (a human-readable data serialization standard, commonly used for configuration files) data file in order to aid parameter fine-tuning.

Machine learning (ML) is a system that can automatically produce a desired output from a given input. ML could learn and improve its decision making from the many correct input-output examples that are given in the training phase. Unlike classic algorithms and programs, ML does not require manually programming the desired response for all possible inputs. It could extract features and recognize general patterns from highly variant data and then find the non-linear mapping from input to output. Deep Neural Networks (DNNs) \citep{LeCun2015} is a well-developed approach to ML. There is no need to pre-compute and extract features based on prior knowledge about the data. Feeding a DNN model with minimally pre-processed data would be sufficient for it to learn and then produce reliable results. With these advantages, we could build an ML system to classify breast cancer tumours with minimal human input. Such a system could improve the efficiency and accuracy in breast cancer diagnostics.

Specifically, the deep learning models used in this project are Convolutional Neural Networks (CNNs). CNN \citep{Krizhevsky2017} is an advanced architecture of DNN. Convolutional layers allow a CNN to learn models that have scale and translational invariance. A convolutional layer convolves filters across an image in order to localize and synthesize important information. Pooling layers are to subsample images by pooling information from patches of pixels between convolution layers. The structure of CNN can effectively build a hierarchy of features progressing from local to global in scope. This unique feature of a CNN makes it well-suited for our complex US and PAT image classification tasks.

This report is organized as follows. In Chapter \ref{background}, we introduce the basic principles of fully-connected neural networks and convolutional neural networks. Chapter \ref{cnn_architectures} presents three CNN architectures used in this project. Chapter \ref{experiments} shows our experimental results. Finally, Chapter \ref{conclusion} is the conclusion.

