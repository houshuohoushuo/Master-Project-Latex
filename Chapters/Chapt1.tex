\chapter{Introduction} % Main chapter title

\DeclareAcronym{dnn}{
	short = DNN,
	long  = Deep Neural Network
}

\DeclareAcronym{cnn}{
	short = CNN,
	long  = Convolutional Neural Network
}


The complex structure of tissue and highly variable image produced by US and PAT make cancer classification a challenging problem. With the recent development of machine learning algorithms, we can make use of the large amounts of data we already have to help clinicians to make more accurate diagnoses. 

Generally speaking, Machine learning (ML) is a system that can automatically produce the desired output from a given input. ML, as the name suggests, could learn and improve its decision making from the many correct input-output examples that its given in the training phase. Unlike classic algorithms and programs, ML does not require manually program the desired response for all possible inputs. ML benefits from large amount of data. It could extract features and recognize general patterns from highly variant data then find the non-linear mapping from input to the correct output. 

Support Vector Machine (SVM) is a well-known classification algorithm that does decision-making. In many previous studies related to clinical image diagnostic, SVM has been trained to produce satisfactory results. However, SVM relies on carefully hand-crafted features. One with no domain knowledge of the original data could not generalize and reuse the model on new data.

\ac{dnn}, on the other hand, is a newly developed approach that works on raw data. There is no need to precompute and extract features that based on prior knowledge about the data. Feeding a \ac{dnn} model with minimally pre-processed data would be sufficient enough for it to learn then produce reliable results. \acp{dnn} are extremely capable of modelling highly complex data that contain hierarchical information. For instance, a \ac{dnn} could be trained to recognize faces. It does so by learning a hierarchy of visual features that distinguish faces from other objects. Starting from lower order features such as edges and corners, to features resembling parts of faces such as eyes and mouth, then finally features that are recognizable as template faces.

\ac{cnn} is an advanced architecture of \ac{dnn}. Its convolution layers allow \ac{cnn} to learn models that have scale and translational invariance. Convolution layer is a \ac{dnn} layer that convolves filters across an image in order to localize and synthesize important information. In a \ac{cnn}, there are pooling layers work together with convolution layers. Pooling layer is to subsample images by pooling information from patches of pixels between convolutional layers. It can effectively build a hierarchy of features progressing from more local to more global in scope. This unique feature of \ac{cnn} makes it well-suited for complex image classification tasks.

The primary objective of this project is to classify breast cancer specimen using CNN. Three CNN models with state-of-the-art architecture are trained on PAT and US images separately. Their classification performance is tested via k-fold cross validation. This project also parameterized the models in YAML format in order to aid parameter fine tuning.





